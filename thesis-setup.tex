% !TEX root = thesis.tex


% **************************************************
% Files' Character Encoding
% **************************************************
\PassOptionsToPackage{utf8}{inputenc}
\usepackage{inputenc}


% Information and Commands for Reuse
% **************************************************
\newcommand{\thesisTitle}{An Asynchronous Evaluation Engine for Stream Based Specifications}
\newcommand{\thesisTitleGerman}{Asynchrone Evaluierung von Strombasierten Spezifikation}
\newcommand{\thesisName}{Alexander Schramm}
\newcommand{\thesisSubject}{Masterarbeit}
\newcommand{\thesisDate}{20. November 2016}
\newcommand{\thesisVersion}{My First Draft}

\newcommand{\thesisFirstReviewer}{Prof.\ Dr.\ Martin Leucker}
\newcommand{\thesisFirstReviewerUniversity}{\protect{University of Luebeck}}
\newcommand{\thesisFirstReviewerDepartment}{Institute For Software Engineering and Programming Languages}

% \newcommand{\thesisSecondReviewer}{Who Knöws}
% \newcommand{\thesisSecondReviewerUniversity}{\protect{University of Luebeck}}
% \newcommand{\thesisSecondReviewerDepartment}{We will see}

\newcommand{\thesisFirstSupervisor}{Cesar Sanchez, Ph.D.}

\newcommand{\thesisAffiliation}{IMDEA Software Institute}

\newcommand{\thesisUniversity}{\protect{University of Luebeck}}
% \newcommand{\thesisUniversityDepartment}{Institute For Software Engineering and Programming Languages}
\newcommand{\thesisUniversityInstitute}{Institute For Software Engineering and Programming Languages}
% \newcommand{\thesisUniversityGroup}{Clean Thesis Group (CTG)}
\newcommand{\thesisUniversityCity}{Luebeck}
\newcommand{\thesisUniversityStreetAddress}{Ratzeburger Allee 160}
\newcommand{\thesisUniversityPostalCode}{23562}

% **************************************************
% Load and Configure Packages
% **************************************************
\usepackage[english]{babel} % babel system, adjust the language of the content
\PassOptionsToPackage{% setup clean thesis style
  figuresep=colon,%
  sansserif=false,%
  hangfigurecaption=false,%
  hangsection=true,%
  hangsubsection=true,%
  colorize=full,%
  colortheme=bluemagenta,%
  bibsys=biber,%
  bibfile=bib-refs,%
  bibstyle=numeric,%
  wrapfooter=false,%
}{cleanthesis}
\usepackage{cleanthesis}

\hypersetup{% setup the hyperref-package options
  pdftitle={\thesisTitle},    %   - title (PDF meta)
  pdfsubject={\thesisSubject},%   - subject (PDF meta)
  pdfauthor={\thesisName},    %   - author (PDF meta)
  plainpages=false,           %   -
  colorlinks=false,           %   - colorize links?
  pdfborder={0 0 0},          %   -
  breaklinks=true,            %   - allow line break inside links
  bookmarksnumbered=true,     %
  bookmarksopen=true          %
}
\usepackage{amssymb}
\usepackage{amsmath}
\usepackage{amsthm}
\usepackage{thmtools}
\usepackage{lscape}
\usepackage[toc]{glossaries}
\loadglsentries{glossary}
\makeglossaries
\usepackage{cleveref}
\crefname{enumi}{case}{cases}
\Crefname{Enumi}{Case}{Cases}
\crefname{lstlisting}{listing}{listings}
\Crefname{Lstlisting}{Listing}{Listings}
\usepackage{multicol}
\usepackage{tabularx}
\usepackage{longtable}
\usepackage{booktabs}
\usepackage{tikz}
\usepackage{tikz-timing}
\usetikzlibrary{arrows, arrows.meta,
  decorations, decorations.pathreplacing, decorations.pathmorphing,
  shapes.misc,
  automata,
  backgrounds,
  positioning,
  fit,
  petri
}
\tikzset{ampersand replacement=\&, % alternative to using fragile frames
  event/.style = {draw, cross out, inner sep = 1pt,xshift=0.75pt},
  event label/.style = {font = \tiny, label distance = -2pt}
}
\newcommand{\n}[1]{N[event, label={[event label]above:#1}]}

\usepackage{listings}
\lstset{
  numbers=left,
  stepnumber=3,
  numbersep=5pt,
  numberstyle=\small\color{black},
  basicstyle=\ttfamily,
  keywordstyle=\color{ctcolormain},
  commentstyle=\color{black},
  stringstyle=\color{ctcoloraccessory},
  tabsize=2,
  captionpos=b,
  frame=top,frame=bottom,
  aboveskip=20pt,
}

\usepackage{color}
\cthesissetcolor{RGB}{0,75,90}{160,187,47}
\addbibresource{bib-refs}

\declaretheoremstyle[
  notefont=\bfseries, notebraces={}{},
  bodyfont=\normalfont\itshape,
  headformat=\NAME~\NUMBER:\NOTE,
  postheadspace = \newline
]{default}
\declaretheoremstyle[
  notefont=\bfseries, notebraces={}{},
  bodyfont=\normalfont\itshape,
  headformat=\NAME~\NUMBER:,
  postheadspace = \newline
]{example}

\declaretheorem[style=default]{lemma}
\declaretheorem[style=default]{definition}
\declaretheorem[style=default]{theorem}
\declaretheorem[style=default, name = Example]{exmp}

\newcommand{\defeq}{\mathrel{\texttt{:=}}}

\newcommand{\dotrel}[1]{\mathrel{\dot{#1}}}

\newcommand{\textttbf}[1]{\texttt{\textbf{#1}}}

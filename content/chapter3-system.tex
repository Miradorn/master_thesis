% !TEX root = ../thesis-example.tex
%
\chapter{System}
\label{sec:system}

Let $E$ be the Set of valid input events and $e \in E$, where each event carries a value, a timestamp and the
channel it is perceived on (e.g.\ a function call of a specific function).
Further let $O$ be the Set of valid output events and $o \in O$, which have the same properties than input events.

% \cleanchapterquote{Innovation distinguishes between a leader and a follower.}{Steve Jobs}{(CEO Apple Inc.)}
\section{Semantics of TeSSLa functions}
Think of semantics of thinks like add, mrv, etc.
Can these be defined independent of async/synchronous or for each.
For synchronous they are partly defined in the TeSSLa spec

\section{Semantics of an ideal, synchronous evaluation Engine}
\label{sec:system:semantics_ideal}
An ideal implementation $I$ for a specification $T$ is one that consumes an input event $e_x$ and immediately emits
appropriate output Events $o_{1,1}, o_{1,2}, \dots , o_{1,x}$ with $x \in \mathbb{N}_{\ge0}$ and changes it's internal
state $S$ to save the fact, that the event $e_x$ was received and optionally other properties generated by the
evaluation of the spec that are needed by later computations.

Therefore it's behaviour for a single input can be described by two functions:
\begin{align*}
    \Phi&: S \times E \rightarrow S \\
    \Theta&: S \times E \rightarrow O^*
\end{align*}

The behaviour for multiple inputs is the composition of the functions.

The semantics for the implementation $I$ for a Stream of input events from $E^*$ is given by the output Stream
from $O^*$ that is generated by the formulas.

\section{Semantics of an asynchronous evaluation Engine}
\label{sec:system:semantics_async}

An asynchronous implementation $A$ for a specification $T$ has more complex characteristics:
It's state is defined by the product of the States of it's nodes, where each node represents a primitive operation in
the specification and the nodes are organized as a DAG.%TODO glossary
The output is specified by the concatination of outputs of the nodes that are marked as output nodes in the specification.

In contrast to $I$ the asynchronous implementation takes multiple steps to produce an output from an input.
During a step multiple things can happen at once:
\begin{itemize}
    \item A new, external input Event can be consumed by a source in the DAG and is propagated to it's children
    \item Multiple internal Nodes, which have at least one new input buffered on their input queue can perform
        their computation and propagete their new output to their children.
    \item Multiple output nodes, which have at least one new input buffered on their input queue, can produce a
        new output.
\end{itemize}

The semantic of the implementation is given by the product of the semantics of it's nodes.

\section{Equality of Semantics}
$A$ and $S$ are considered equal if for a Series of input Events they produce the same output Events.
$A$ won't emit the outputs in the right order, but because each event holds the timestamp it was generated, they can
be reordered so that the output will be exactly the same as the one from $I$

\subsection{Equality of States}
To proof the equality of both systems we have to proof the equality of the states $A$ will take wile producing the
outputs to the States $I$ takes.
Let $e_1, e_2, \dots, e_x$ be the input events both implementations receive.
The State of $I$ will change every time a new input ic received and will produce all outputs that could be computed.
It can be visualized as:
\noindent
\begin{flalign*}
    &\text{Input:}        &&e_1\                            &&e_2                        &&e_3 &\\
    &\text{new State:}    &&I_1\                            &&I_2                        &&I_3 &\\
    &\text{new Outputs:}  &&(o_{1,1}, \dots, o_{1,x})    \   &&(o_{2,1}, \dots, o_{2,y})  &&(o_{3,1}, \dots, o_{3,y}) &
\end{flalign*}
% \begin{figure}[htb]
%     \includegraphics[width=\textwidth]{gfx/Clean-Thesis-Figure}
%     \caption{Figure example: \textit{(a)} example part one, \textit{(c)} example part two; \textit{(c)} example part three}
%     \label{fig:system:example1}
% \end{figure}

% \begin{figure}[htb]
%     \includegraphics[width=\textwidth]{gfx w/Clean-Thesis-Figure}
%     \caption{Another Figure example: \textit{(a)} example part one, \textit{(c)} example part two; \textit{(c)} example part three}
%     \label{fig:system:example2}
% \end{figure}


% !TEX root = ../thesis-example.tex
%
\chapter{Related Work}
\label{sec:related}

% \cleanchapterquote{A picture is worth a thousand words. An interface is worth a thousand pictures.}{Ben Shneiderman}{(Professor for Computer Science)}
As Runtime Verification is a widely researched field there are many different approaches towards monitoring programs.
TeSSLa itself and the implemented runtime builds on concepts and results of many of them.
In the following section some of them are highlighted to give a better understanding of choices made during this thesis.

\section{LOLA}
\label{sec:related:lola}

LOLA~\cite{DAngelo2005} may arguably have the biggest influence on TeSSLa and the theoretical work of this thesis.
LOLA defines a very small core language to describe streams as the result of combinations of other streams of events.
In contrast to TeSSLa it is defined in regards of a discrete timing model.

Lola defines a notion of efficiently monitorable properties and an approach to monitor these properties.

TeSSLa takes concepts of LOLA and applies them to a continuous model of time and introduces a language and a rich
set of functions that can be applies to streams.

\section{Copilot}
\label{sec:related:copilot}

The realtime runtime monitor system Copilot was introduced in~\cite{Pike2010}.
Copilot is designed to overcome the shortcomings of existing RV tools in regards to hard-realtime software written in C.

To do so they first define characteristics a monitoring approach has to fullfill to be considered valuable for this domain.
The four principles are:

\begin{description}
  \item[Functionality] Monitors cannot change the functionality of the observed program unless a failure is observed.
  \item[Schedulability] Monitors cannot alter the schedule of the observed program.
  \item[Certifiability] Monitors must minimize the difficulty in re-validating the observed program; in particular, we make it our goal to avoid modifying the observed programs source code.
  \item[SWaP overhead] Monitors must minimize the additional overhead required including size, weight, and power (SWaP).
\end{description}

The monitors follow a sampling based approach, where at specified steps the values of global variables are observed and the monitors are evaluated
on that values.
While sampling based approaches are widely disregarded in RV, because they can lead to both false positives and false negatives,
they argue:

\begin{quote}
  In a hard real-time context, sampling is a suitable strategy. Under
  the assumption that the monitor and the observed program share a global clock and a static periodic schedule, while false positives are possible, false negatives are not.~\cite{Pike2010}
\end{quote}

A special detail of Copilot is that monitors aren't inlined into the program but can be scheduled as independet processes.
The implementation of the TeSSLa runtime in this thesis follows a similar approach: It is a totally independent program,
and therefore also has some of the gains in regard to the specified four characteristics.
Because the runtime works with all kinds of traces, it is insignificant how they are produced:
It can work with traces based on sampling, working in a similar fashion as Copilot, or by actually instrumenting code to generate
traces, which alters the semantics of the program.

% Another shortcoming of inlining monitors is that certified code (e.g., DO-178B for avionics [Inc92]) is common in this domain.
% Inlining monitors could necessitate re-certifying the observed program.

\section{RMoR}
\label{sec:related:rmor}

\section{Driver Trace}
\section{Debie}
\section{MaC}
\section{RiTHM}



% !TEX root = ../thesis-example.tex
%
\chapter{Introduction}
\label{sec:intro}

% \cleanchapterquote{You can’t do better design with a computer, but you can speed up your work enormously.}{Wim Crouwel}{(Graphic designer and typographer)}


\section{Motivation and Problem Statement}
\label{sec:intro:motivation}

Program verification is a field of computer sience that is widely researched and has multiple approaches.
While pure verification approaches try to proof the correct behaviour of a system under all possible executions,
Runtime Verifications limits itself to single, finite runs of a system, trying to proof it conforms to a given specification
under specific conditions, like input sequences or scheduling.
These specifications can be given in various ways, like LTL formulas or in specification languages that are specifically developed for
runtime verification.
A specific field of RV is looking at verifying behviour of streams of data, specifying relationships of values on those streams.
Examples for this are RMoR, Lola and TeSSLa, which we will look at more closely in Section~\ref{sec:related}.

The language TeSSLa aims to make it easy to specify behaviour of streams.
To gain this it introduces a number of language features and sintax sugare to expressively describe the conditions a stream should fulfill.
The evaluation of TeSSLa specifications is done in two steps: first the specification is compiled by a compiler written at the ISP of the University of Luebeck.
The output is a canonical representation of the operations on the streams in the specification.
In the second step the compiled specification is connected with a system that produces some kind of traces, which are treated as the input streams of the specification.

The second step can be done in different ways: online or offline, interweaving the monitors into the monitored program (see RMoR) or having a standalone system.
At the beginning of this thesis there was one implementation of a runtime for this monitor that is based on FPGAs that have to be manually reconfigured for each new specification.
While this is a very performant approach for the actual monitoring it isn't usable for testing and prototyping.
Therefore it is wanted to implement a runtime for TeSSLa specifications that can be run independent of specific hardware.

During the Thesis it is proofen that the actual approach of this runtime, a functional, actor based, asynchronous system,
will generate the same observations on input traces as an ideal evaluation of the specification.

\section{Results}
\label{sec:intro:results}


\section{Thesis Structure}
\label{sec:intro:structure}

\textbf{Chapter~\ref{sec:related}} \\[0.2em]

\textbf{Chapter~\ref{sec:system}} \\[0.2em]

\textbf{Chapter~\ref{sec:concepts}} \\[0.2em]

\textbf{Chapter~\ref{sec:concepts}} \\[0.2em]

\textbf{Chapter~\ref{sec:conclusion}} \\[0.2em]

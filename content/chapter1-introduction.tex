% !TEX root = ../thesis.tex
%
\chapter{Introduction}
\label{sec:intro}

% \cleanchapterquote{You can’t do better design with a computer, but you can speed up your work enormously.}{Wim Crouwel}{(Graphic designer and typographer)}

\section{Motivation and Problem Statement}
\label{sec:intro:motivation}

Software verification is an important tool to harden critical systems against faults and exploits.
Due to the raising importance of computer based systems, verification has become a big field of research in computer science.

While pure verification approaches try to proof the correct behaviour of a system under all possible executions, \gls{rv} limits itself to single, finite runs of a system.
The goal is to proof that a run conforms to a given specification under specific conditions, like input sequences or scheduling.
Specifications can be given in various ways, including \gls{tl} formulas or in specification languages that are specifically developed for \gls{rv}.
Examples for this are \gls{rmor} \citep{Havelund2008}, \gls{lola} \citep{DAngelo2005} and others \citep{Zheng2015, Pike2010, Mostafa2015}, which we will look at more closely in \Cref{sec:related}.

The project \Gls{tessla}\citep{Decker2016} presents ways to specify and evaluate properties over streams of events including timing information.
To achieve this it introduces a language to expressively describe the conditions one or more streams should fulfill by applying transformations on them.
The evaluation of \gls{tessla} a specification is done in two steps: first the specification is compiled by a compiler written at the \gls{isp} of the University of Lübeck.
The output is a canonical representation of the transformations on the streams in the specification.
In the second step the compiled specification is connected with a system that produces traces that are treated as the input streams of the specification.

The second step can be done in different ways: online or offline, interweaving the monitors into the monitored program (like for example done in~\cite{Havelund2008}) or by executing them standalone.
These different approaches lead to different ways the monitored program has to be altered, for example manipulating its original code to log status informations or to invoke the monitoring code.

Interweaved monitors can alter the original system and produce new errors or even suppress others.
Standalone monitors on the other hand will have a much smaller impact on the monitored system.
But as a consequence there will be a bigger delay between the occurence of events in the program and their evaluation in the monitor.
Furthermore interweaved monitors can optionally react to detected errors.
They could change the control flow of the original system or alert a third party and eliminate cascading errors.
Standalone monitors can't directly modify the program but can still produce warnings and alerts that can then be reacted to.

While online monitoring can be used to actively react to error conditions, either automatically or by notification of a third party, offline monitoring can be thought of as an extension to software testing~\cite{DAngelo2005}.

At the beginning of this thesis there was one implementation of a runtime for \gls{tessla} specifications that is based on \glspl{fpga} that have to be manually reconfigured for each new specification.
While this is a very performant approach for actual monitoring it isn't feasible for testing and prototyping.
This leads to the wish to implement a software based \gls{tessla} runtime which can be executed independent of hardware restrictions.

Furthermore most \gls{rv} approaches are specific to one programming language or environment and combine ways of generating the data, which is used for monitoring, and the monitoring itself.
\gls{tessla} specifications themself are independent of any implementation details of the monitored system, working only on streams of data, which can be gathered in any way.
This can be used to implement a runtime that is also independent of the monitored system an how traces of it are collected.

During the thesis it is prooven that the actual approach of the runtime, a functional, actor based, asynchronous system, will generate the same observations on input traces as a synchronous evaluation of the specification.
While \gls{tessla} specifications can work on all kinds of streams, especially on traces on all levels of a program, including instruction counters or spawning processes, in this thesis we will mainly focus on the level of function calls and variable reads/writes.
Other applications of the system could easily extend it to use traces representing drastically different fields, for example health data, temperatures, battery levels, web services and more.

To test the software based runtime different specifications will be tested on multiple traces.
Some of the traces are generated by actually running a program which was instrumented by hand or automatically to generate traces.
Others where are generated or modified by hand to deliberately introduce bugs which should be detected by the system.

\section{Results}
\label{sec:intro:results}

The main contributions of this thesis consist of three parts.
The first is a theoretical approach to asynchronously evaluate timed specifications over streams.
The second is an implementation that can synthesize systems to evaluate such specificationss based on the theoretical approach.
And the third is a proof of concept implementation of a system that can instrument code which is compiled with \gls{llvm}, mainly targeted at C and C++.
TODO: more


\section{Thesis Structure}
\label{sec:intro:structure}

As the whole evaluation engine is built on top of different technical and theroetical ideas, it is structured to show
the reasoning behind the decisions that were made during the development.
Furthermore it will proof equalitys of different kinds of systems in multiple steps that build on one another.
In the following a quick overview of the different parts of the thesis is given.

\textbf{\Cref{sec:related}}

In this chapter the theoretical foundation for the system is explained.
Furthermore multiple approaches solving similar problems are shown and it is highlighted which concepts of them were
used in the new system and which were disregarded and why.


\textbf{\Cref{sec:definitions}}

Building on the theoretical and practical findings of the previous chapters new definitions are presented, which are needed to reason about the implemented system.

\textbf{\Cref{sec:behaviours}}

The work from the previous chapter is put to work to reason about the semantics of the implemented runtime and to show its correctness.

\textbf{\Cref{sec:implementation}}

This chapter highlights technical details of the implementation.
It will present alternative implementation approaches and the reasoning why specific choices were made during the development of the system.

\textbf{\Cref{sec:evaluation}}

To show the value of the implemented system it is thoroughly tested and benchmarked with fabricated and real world examples and traces.
The results of this testing is used to evaluate the implementation.

\textbf{\Cref{sec:conclusion}}

On the basis of the evaluation in the conclusion the results of the thesis are summarized.
Furthermore it is highlighted what remains to do and which future challenges exist.



% !TEX root = ../thesis.tex
%
\pdfbookmark[0]{Abstract}{Abstract}
\chapter*{Abstract}
\label{sec:abstract}
\vspace*{-10mm}

This thesis studies the problem of software reliability using monitors specified with a stream runtime verification language.
In particular, we study the problem of evaluating specifications against finite streams of data.
The specifications we consider are written in the \gls{tessla} specification language, come from the field of Runtime Verification and describe correct behavior of running software systems.

Whether a run of a given system is correct is evaluated over trace data that is collected while the system is executing.
This data trace is represented as a collection of streams.
The specification states that this collection of input streams must fulfill specified conditions.

The first contribution of this thesis is the implementation of a \gls{tessla} evaluation engine, using an asynchronous and distributed approach to combine streams.
The engines we propose can check whether the traces produced by the running system satisfy the given specification.
The asynchronous nature of the engines we propose allows our solutions to scale to several parallel execution components for the evaluation engine.

The second contribution of this thesis is a proof of correctness of the implemented engines, based on the possible execution orders between the parallel asynchronous components.
We show that even the most asynchronous implementation produces the same verdicts as the ideal fully synchronize engine.



\clearpage

{\usekomafont{chapter}Abstract (Deutsch)}\label{sec:abstract-diff} \\

Diese Arbeit untersucht das Problem der Softwarezuverlässigkeit unter Nutzung von Monitoren, die mit einer strombasierten Runtime Verification sprache spezifiziert werden.
Im speziellen untersuchen wir die Auswertung von Spezifikationen über endliche Datenströme.
Die berücksichtigten Spezifikationen sind in der \gls{tessla} spezifikationssprache geschrieben, kommen aus dem Feld der Runtime Verification und beschreiben korrektes Verhalten von laufenden Softwaresystemen.

Ob ein Lauf eines gegebenen Systems korrekt ist wird über Tracedaten ausgewertet, die während der Ausführung des Programms gesammelt werden.
Diese Tracedaten werden als eine Menge von Datenströmen repräsentiert.
Eine Spezifikation verlangt, dass diese Menge von Datenströmen gegebene Bedingungen erfüllt.

Das erste Ergebnis dieser Arbeit ist die Implementierung eines \gls{tessla} Evaluierungs\-systems, welches einen asynchronen, verteilten Ansatz zur Kombination von Datenströmen benutzt.
Das System, welches wir vorstellen, ist in der Lage zu erkennen, ob die Tracedaten eines laufenden Systems eine Spezifikation erfüllen.
Die asynchrone Natur des Systems erlaubt es, das Evaluierungssystem auf mehrere, parallel ausführende Recheneinheiten zu skalieren.

Ein zweites Ergebnis dieser Arbeit ist ein Beweis der Korrektheit des implementierten Systems, basierend auf den möglichen Ausführungsreihenfolgen der parallel laufenden komponenten.
Wir zeigen, dass selbst ein maximal asynchron ausgeführtes System zu demselben Urteil, wie ein ideales, synchrones System, kommt.
